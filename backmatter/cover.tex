\cleardoublepage
\begingroup
\thispagestyle{empty}
\begin{tikzpicture}[remember picture,overlay]
  \node[inner sep=0pt] (background) at (current page.center) {%
    \ifusechapterimage
      \reflectbox{%
        \includegraphics[width=\paperwidth,height=\paperheight]{cover.jpg}%
      }%
    \fi
  };
  \draw (current page.center) node%
    [fill=coverblue!30!white,fill opacity=0.6,text opacity=1,inner sep=1cm] {%
    \parbox[c][][t]{\paperwidth - 1cm * 2} {%
      \normalsize\sffamily
      When I was a high school student on FIRST Robotics Competition (FRC) team
      3512, I had to learn control theory from scattered internet sources that
      either weren't rigorous enough or assumed too much prior knowledge. After
      I took graduate-level control theory courses from University of
      California, Santa Cruz for my bachelor's degree, I realized that the
      concepts weren't difficult when presented well, but the information wasn't
      broadly accessible outside academia. \\

      I wanted to fix the information disparity so more people could appreciate
      the beauty and elegance I saw in control theory. This book streamlines the
      learning process to make that possible. \\

      I wrote the initial draft of this book as a final project for an
      undergraduate technical writing class I took at UCSC in Spring 2017 (CMPE
      185). It was a 13-page IEEE-formatted paper intended as a reference manual
      and guide to state-space control that summarized the three graduate
      controls classes I had taken that year. I kept working on it the following
      year to flesh it out, and it eventually became long enough to make into a
      proper book. I've been adding to it ever since as I learn new things. \\

      I contextualized the material within FRC because it's always been a
      significant part of my life, and it's a useful application sandbox. I
      maintain implementations of many of this book's tools in the FRC standard
      library (WPILib). \\

      -- Tyler Veness
    }%
  };
\end{tikzpicture}
\endgroup
