\chapterimage{linear-algebra.jpg}{Grass clearing by Interdisciplinary Sciences building and Thimann Labs at UCSC}

\chapter{Linear algebra}

Modern control theory borrows concepts from linear algebra. At first, linear
algebra may appear very abstract, but there are simple geometric intuitions
underlying it. We'll be linking to 3Blue1Brown's
\href{https://www.3blue1brown.com/topics/linear-algebra}{\textit{Essence of
linear algebra}} video series because it's better at conveying that intuition
than static text.
\begin{bookfigure}
  \qrcode{https://www.3blue1brown.com/lessons/eola-preview} \\
  ``Essence of linear algebra preview" (5 minutes) \\
  \footnotesize 3Blue1Brown \\
  \url{https://www.3blue1brown.com/lessons/eola-preview}
\end{bookfigure}

\renewcommand*{\chapterpath}{\partpath/linear-algebra}
\section{Vectors}
\index{linear algebra!vectors}
\begin{bookfigure}
  \qrcode{https://www.3blue1brown.com/lessons/vectors} \\
  ``Vectors, what even are they?" (5 minutes) \\
  \footnotesize 3Blue1Brown \\
  \url{https://www.3blue1brown.com/lessons/vectors}
\end{bookfigure}

\section{Linear combinations, span, and basis vectors}
\index{linear algebra!basis vectors}
\index{linear algebra!linear combination}

Watch the ``Linear combination, span, and basis vectors" video from
3Blue1Brown's \textit{Essence of linear algebra} series (10 minutes) or read the
associated lesson on 3blue1brown.com
\cite{bib:3b1b_linalg_linear_combinations}.
\begin{bookfigure}
  \qrcode{https://www.youtube.com/watch?v=k7RM-ot2NWY} \\
  \tiny
  ``Linear combination, span, and basis vectors" \\
  \url{https://www.youtube.com/watch?v=k7RM-ot2NWY}
\end{bookfigure}

\section{Linear transformations and matrices}
\index{matrices!linear transformation}

Watch the ``Linear transformations and matrices" video from 3Blue1Brown's
\textit{Essence of linear algebra} series (11 minutes) or read the associated
lesson on 3blue1brown.com
\cite{bib:3b1b_linalg_linear_transformations_and_matrices}.
\begin{bookfigure}
  \qrcode{https://www.youtube.com/watch?v=kYB8IZa5AuE} \\
  \tiny
  ``Linear transformations and matrices" \\
  \url{https://www.youtube.com/watch?v=kYB8IZa5AuE}
\end{bookfigure}

\section{Matrix multiplication}
\index{matrices!multiplication}

Watch the ``Matrix multiplication as composition" video from 3Blue1Brown's
\textit{Essence of linear algebra} series (10 minutes) or read the associated
lesson on 3blue1brown.com
\cite{bib:3b1b_linalg_matrix_multiplication_as_composition}.

\section{The determinant}
\index{matrices!determinant}
\begin{bookfigure}
  \qrcode{https://www.3blue1brown.com/lessons/determinant} \\
  ``The determinant" (10 minutes) \\
  \footnotesize 3Blue1Brown \\
  \url{https://www.3blue1brown.com/lessons/determinant}
\end{bookfigure}

\section{Inverse matrices, column space, and null space}
\index{matrices!linear systems}
\index{matrices!inverse}
\index{matrices!rank}
\begin{bookfigure}
  \qrcode{https://www.3blue1brown.com/lessons/inverse-matrices} \\
  ``Inverse matrices, column space, and null space" (12 minutes) \\
  \footnotesize 3Blue1Brown \\
  \url{https://www.3blue1brown.com/lessons/inverse-matrices}
\end{bookfigure}

\section{Nonsquare matrices}
\begin{bookfigure}
  \qrcode{https://www.3blue1brown.com/lessons/nonsquare-matrices} \\
  ``Nonsquare matrices as transformations between dimensions" (4 minutes) \\
  \footnotesize 3Blue1Brown \\
  \url{https://www.3blue1brown.com/lessons/nonsquare-matrices}
\end{bookfigure}

\section{Eigenvectors and eigenvalues}
\index{matrices!eigenvalues}
\begin{bookfigure}
  \qrcode{https://www.3blue1brown.com/lessons/eigenvalues} \\
  ``Eigenvectors and eigenvalues" (17 minutes) \\
  \footnotesize 3Blue1Brown \\
  \url{https://www.3blue1brown.com/lessons/eigenvalues}
\end{bookfigure}

\section{Miscellaneous notation and operators}

This book works with two-dimensional matrices in the sense that they only have
rows and columns. The dimensionality of these matrices is specified by row
first, then column. For example, a matrix with two rows and three columns would
be a two-by-three matrix. A square matrix has the same number of rows as
columns. Matrices commonly use capital letters while vectors use lowercase
letters.

\subsection{Special constant matrices}

$\mat{I}$ is the identity matrix, a typically square matrix with ones along its
diagonal and zeroes elsewhere. $\mat{0}$ is a matrix filled with zeroes and
$\mat{1}$ is a matrix filled with ones. An optional subscript ${m \times n}$
denotes the matrix having $m$ rows and $n$ columns.
\begin{equation*}
  \mat{I}_{3 \times 3} =
  \begin{bmatrix}
    1 & 0 & 0 \\
    0 & 1 & 0 \\
    0 & 0 & 1
  \end{bmatrix}
  \quad
  \mat{0}_{3 \times 2} =
  \begin{bmatrix}
    0 & 0 \\
    0 & 0 \\
    0 & 0
  \end{bmatrix}
  \quad
  \mat{1}_{3 \times 2} =
  \begin{bmatrix}
    1 & 1 \\
    1 & 1 \\
    1 & 1
  \end{bmatrix}
\end{equation*}

\subsection{Operators}

\subsubsection{Transpose}
\index{matrices!transpose}
The $\T$ in $\mat{A}\T$ denotes transpose, which flips the matrix across its
diagonal such that the rows become columns and vice versa.

A symmetric matrix is equal to its transpose.

\subsubsection{Pseudoinverse}
\index{matrices!pseudoinverse}
The $^+$ in $\mat{B}^+$ denotes the Moore-Penrose pseudoinverse given by
$\mat{B}^+ = (\mat{B}\T\mat{B})^{-1}\mat{B}\T$. The pseudoinverse is used when
the matrix is nonsquare and thus not invertible to produce a close approximation
of an inverse in the least squares sense.

\subsubsection{Diagonal}
\index{matrices!diagonal}
A diagonal matrix has elements along its diagonal and zeroes elsewhere (e.g.,
the identity matrix). Let
$\mat{x} = \begin{bmatrix}x_1 & \ldots & x_n\end{bmatrix}\T$.
\begin{equation*}
  \diag(\mat{x}) = \diag(x_1,\, \ldots,\, x_n) =
  \begin{bmatrix}
    x_1 & 0 & \cdots & 0 \\
    0 & x_2 & & \vdots \\
    \vdots & & \ddots & 0 \\
    0 & \cdots & 0 & x_n
  \end{bmatrix}
\end{equation*}

A block diagonal matrix has matrices along its diagonal. $\diag()$ works
similarly for constructing one. Let
$\mat{A} = \begin{bsmallmatrix}1 & 2\\3 & 4\end{bsmallmatrix}$ and
$\mat{B} = \begin{bsmallmatrix}1 & 2 & 3\\4 & 5 & 6\\7 & 8 & 9\end{bsmallmatrix}$.
\begin{equation*}
  \diag(\mat{A}, \mat{B}) =
  \begin{bmatrix}
    \mat{A} & \mat{0} \\
    \mat{0} & \mat{B}
  \end{bmatrix} =
  \begin{bmatrix}
    1 & 2 & 0 & 0 & 0 \\
    3 & 4 & 0 & 0 & 0 \\
    0 & 0 & 1 & 2 & 3 \\
    0 & 0 & 4 & 5 & 6 \\
    0 & 0 & 7 & 8 & 9
  \end{bmatrix}
\end{equation*}

Operations on the $\diag()$ argument are applied element-wise.
\begin{equation*}
  \diag\left(\frac{1}{\mat{x}^2}\right) =
  \begin{bmatrix}
    \frac{1}{x_1^2} & 0 & \cdots & 0 \\
    0 & \frac{1}{x_2^2} & & \vdots \\
    \vdots & & \ddots & 0 \\
    0 & \cdots & 0 & \frac{1}{x_n^2}
  \end{bmatrix}
\end{equation*}

\subsubsection{Trace}
\index{matrices!trace}
$\tr(\mat{A})$ denotes the trace of the square matrix $\mat{A}$, which is
defined as the sum of the elements on the main diagonal (top-left to
bottom-right).

\input{\chapterpath/matrix-definiteness}
\section{Common control theory matrix equations}

Here's some common matrix equations from control theory we'll use later on.
Solvers for them exist in \href{https://github.com/RobotLocomotion/drake}{Drake}
(C++) and \href{https://github.com/scipy/scipy}{SciPy} (Python).

\subsection{Continuous algebraic Riccati equation (CARE)}
\index{algebraic Riccati equation!continuous}

The continuous algebraic Riccati equation (CARE) appears in the solution to the
continuous time LQ problem.
\begin{equation}
  \mat{A}\mat{X} + \mat{X}\mat{A} - \mat{X}\mat{B}\mat{R}^{-1}\mat{B}\T\mat{X} +
    \mat{Q} = 0
\end{equation}

\subsection{Discrete algebraic Riccati equation (DARE)}
\index{algebraic Riccati equation!discrete}

The discrete algebraic Riccati equation (DARE) appears in the solution to the
discrete time LQ problem.
\begin{equation}
  \mat{X} = \mat{A}\T\mat{X}\mat{A} - (\mat{A}\T\mat{X}\mat{B})(\mat{R} +
    \mat{B}\T\mat{X}\mat{B})^{-1} \mat{B}\T\mat{X}\mat{A} + \mat{Q}
\end{equation}

Snippet \ref{lst:dare} computes the unique stabilizing solution to the discrete
algebraic Riccati equation. A robust implementation should also enforce the
following preconditions:
\begin{enumerate}
  \item $\mat{Q} = \mat{Q}\T \geq \mat{0}$ and $\mat{R} = \mat{R}\T > \mat{0}$.
  \item $(\mat{A}, \mat{B})$ is a stabilizable pair (see subsection
    \ref{subsec:stabilizability}).
  \item $(\mat{A}, \mat{C})$ is a detectable pair where
    $\mat{Q} = \mat{C}\T\mat{C}$ (see section \ref{subsec:detectability}).
\end{enumerate}
\begin{coderemote}{cpp}{snippets/DARE.cpp}
  \caption{Discrete algebraic Riccati equation solver in C++}
  \label{lst:dare}
\end{coderemote}

\subsection{Continuous Lyapunov equation}
\index{Lyapunov equation!continuous}

The continuous Lyapunov equation appears in controllability/observability
analysis of continuous time systems.
\begin{equation}
  \mat{A}\mat{X} + \mat{X}\mat{A}\T + \mat{Q} = 0
\end{equation}

\subsection{Discrete Lyapunov equation}
\index{Lyapunov equation!discrete}

The discrete Lyapunov equation appears in controllability/observability analysis
of discrete time systems.
\begin{equation}
  \mat{A}\mat{X}\mat{A}\T - \mat{X} + \mat{Q} = 0
\end{equation}

\input{\chapterpath/matrix-calculus}
