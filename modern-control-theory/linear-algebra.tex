\chapterimage{linear-algebra.jpg}{Grass clearing by Interdisciplinary Sciences building and Thimann Labs at UCSC}

\chapter{Linear algebra}

Modern control theory borrows concepts from linear algebra. At first, linear
algebra may appear very abstract, but there are simple geometric intuitions
underlying it. First, watch 3Blue1Brown's preview video for the
\textit{Essence of linear algebra} video series (5 minutes)
\cite{bib:3b1b_linalg_preview}. The goal here is to provide an intuitive,
geometric understanding of linear algebra as a method of linear transformations.
\begin{bookfigure}
  \qrcode{https://www.youtube.com/watch?v=kjBOesZCoqc} \\
  \tiny
  ``Essence of linear algebra preview" \\
  \url{https://www.youtube.com/watch?v=kjBOesZCoqc}
\end{bookfigure}

We would normally include written material here for learning linear algebra, but
3Blue1Brown's animated videos are better at conveying the geometric intuition
involved than anything we could write here with static text. Instead, we'll
provide bibliography entries for lessons on 3blue1brown.com and QR codes for the
corresponding videos.

\renewcommand*{\chapterpath}{\partpath/linear-algebra}
\section{Vectors}
\index{linear algebra!vectors}
\begin{bookfigure}
  \qrcode{https://www.3blue1brown.com/lessons/vectors} \\
  ``Vectors, what even are they?" (5 minutes) \\
  \footnotesize 3Blue1Brown \\
  \url{https://www.3blue1brown.com/lessons/vectors}
\end{bookfigure}

\section{Linear combinations, span, and basis vectors}
\index{linear algebra!basis vectors}
\index{linear algebra!linear combination}

Watch the ``Linear combination, span, and basis vectors" video from
3Blue1Brown's \textit{Essence of linear algebra} series (10 minutes) or read the
associated lesson on 3blue1brown.com
\cite{bib:3b1b_linalg_linear_combinations}.
\begin{bookfigure}
  \qrcode{https://www.youtube.com/watch?v=k7RM-ot2NWY} \\
  \tiny
  ``Linear combination, span, and basis vectors" \\
  \url{https://www.youtube.com/watch?v=k7RM-ot2NWY}
\end{bookfigure}

\section{Linear transformations and matrices}
\index{matrices!linear transformation}

Watch the ``Linear transformations and matrices" video from 3Blue1Brown's
\textit{Essence of linear algebra} series (11 minutes) or read the associated
lesson on 3blue1brown.com
\cite{bib:3b1b_linalg_linear_transformations_and_matrices}.
\begin{bookfigure}
  \qrcode{https://www.youtube.com/watch?v=kYB8IZa5AuE} \\
  \tiny
  ``Linear transformations and matrices" \\
  \url{https://www.youtube.com/watch?v=kYB8IZa5AuE}
\end{bookfigure}

\section{Matrix multiplication}
\index{matrices!multiplication}

Watch the ``Matrix multiplication as composition" video from 3Blue1Brown's
\textit{Essence of linear algebra} series (10 minutes) or read the associated
lesson on 3blue1brown.com
\cite{bib:3b1b_linalg_matrix_multiplication_as_composition}.

\section{The determinant}
\index{matrices!determinant}
\begin{bookfigure}
  \qrcode{https://www.3blue1brown.com/lessons/determinant} \\
  ``The determinant" (10 minutes) \\
  \footnotesize 3Blue1Brown \\
  \url{https://www.3blue1brown.com/lessons/determinant}
\end{bookfigure}

\section{Inverse matrices, column space, and null space}
\index{matrices!linear systems}
\index{matrices!inverse}
\index{matrices!rank}
\begin{bookfigure}
  \qrcode{https://www.3blue1brown.com/lessons/inverse-matrices} \\
  ``Inverse matrices, column space, and null space" (12 minutes) \\
  \footnotesize 3Blue1Brown \\
  \url{https://www.3blue1brown.com/lessons/inverse-matrices}
\end{bookfigure}

\section{Nonsquare matrices}
\begin{bookfigure}
  \qrcode{https://www.3blue1brown.com/lessons/nonsquare-matrices} \\
  ``Nonsquare matrices as transformations between dimensions" (4 minutes) \\
  \footnotesize 3Blue1Brown \\
  \url{https://www.3blue1brown.com/lessons/nonsquare-matrices}
\end{bookfigure}

\section{Eigenvectors and eigenvalues}
\index{matrices!eigenvalues}
\begin{bookfigure}
  \qrcode{https://www.3blue1brown.com/lessons/eigenvalues} \\
  ``Eigenvectors and eigenvalues" (17 minutes) \\
  \footnotesize 3Blue1Brown \\
  \url{https://www.3blue1brown.com/lessons/eigenvalues}
\end{bookfigure}

\input{\chapterpath/misc-notation}
\input{\chapterpath/matrix-definiteness}
\section{Common control theory matrix equations}

Here's some common matrix equations from control theory we'll use later on.
Solvers for them exist in \href{https://github.com/RobotLocomotion/drake}{Drake}
(C++) and \href{https://github.com/scipy/scipy}{SciPy} (Python).

\subsection{Continuous algebraic Riccati equation (CARE)}
\index{algebraic Riccati equation!continuous}

The continuous algebraic Riccati equation (CARE) appears in the solution to the
continuous time LQ problem.
\begin{equation}
  \mat{A}\mat{X} + \mat{X}\mat{A} - \mat{X}\mat{B}\mat{R}^{-1}\mat{B}\T\mat{X} +
    \mat{Q} = 0
\end{equation}

\subsection{Discrete algebraic Riccati equation (DARE)}
\index{algebraic Riccati equation!discrete}

The discrete algebraic Riccati equation (DARE) appears in the solution to the
discrete time LQ problem.
\begin{equation}
  \mat{X} = \mat{A}\T\mat{X}\mat{A} - (\mat{A}\T\mat{X}\mat{B})(\mat{R} +
    \mat{B}\T\mat{X}\mat{B})^{-1} \mat{B}\T\mat{X}\mat{A} + \mat{Q}
\end{equation}

Snippet \ref{lst:dare} computes the unique stabilizing solution to the discrete
algebraic Riccati equation. A robust implementation should also enforce the
following preconditions:
\begin{enumerate}
  \item $\mat{Q} = \mat{Q}\T \geq \mat{0}$ and $\mat{R} = \mat{R}\T > \mat{0}$.
  \item $(\mat{A}, \mat{B})$ is a stabilizable pair (see subsection
    \ref{subsec:stabilizability}).
  \item $(\mat{A}, \mat{C})$ is a detectable pair where
    $\mat{Q} = \mat{C}\T\mat{C}$ (see section \ref{subsec:detectability}).
\end{enumerate}
\begin{coderemote}{cpp}{snippets/DARE.cpp}
  \caption{Discrete algebraic Riccati equation solver in C++}
  \label{lst:dare}
\end{coderemote}

\subsection{Continuous Lyapunov equation}
\index{Lyapunov equation!continuous}

The continuous Lyapunov equation appears in controllability/observability
analysis of continuous time systems.
\begin{equation}
  \mat{A}\mat{X} + \mat{X}\mat{A}\T + \mat{Q} = 0
\end{equation}

\subsection{Discrete Lyapunov equation}
\index{Lyapunov equation!discrete}

The discrete Lyapunov equation appears in controllability/observability analysis
of discrete time systems.
\begin{equation}
  \mat{A}\mat{X}\mat{A}\T - \mat{X} + \mat{Q} = 0
\end{equation}

\input{\chapterpath/matrix-calculus}
